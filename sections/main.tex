\cvsection{Highlights}

\textbf{Seasoned software engineering leader with 15+ years of experience}, spanning roles from \textbf{technical co-founder} to \textbf{CTO} and \textbf{VP of Technology}. Leveraging a strong foundation in \textbf{mathematics and computer science} to design and implement scalable, high-performance systems. Proficient in multiple tech stacks including \textbf{Java, Python, and TypeScript}. Proven track record of fostering \textbf{continuous improvement} and \textbf{collaboration} through agile methodologies.

\vspace{1cm}

\cvsection{Latest Professional Experience}
\cvexpericence{Senior Software Engineer}{Spotify}{(5 years) December 2019 -- Present}{Stockholm, Sweden}
\small
I joined Spotify in 2019, initially as a consultant, before transitioning to a full-time employee after a few months.
\smallskip

My first mission was to \textbf{mentor} and \textbf{lead} a team of engineers in replacing a \textbf{legacy admin dashboard} with new technologies while modernizing and innovating the product. I successfully taught \textbf{TypeScript} and \textbf{React} to the team while creating a new \textbf{microservice-based architecture} using \textbf{async Python}. We innovated by developing a new \textbf{transport protocol} for \textbf{low latency} and \textbf{high reliability}. The product has been in successful use and continuous development for five years, with virtually no tech-related incidents. On the product side, we established close collaboration with our \textbf{stakeholders} and \textbf{users}, conducting numerous \textbf{user studies} and utilizing Spotify's then-new \textbf{design language} to deliver a highly usable product.
\smallskip

During my tenure with the team, I was responsible for defining, prioritizing, and leading \textbf{tech initiatives}. By creating a collaborative setup and running \textbf{tech retrospectives}, I managed to reduce \textbf{technical debt} and maintain high-quality services with low maintenance effort. Some of my achievements include leading and educating the team on the importance of \textbf{SLIs} and \textbf{SLOs} for reliability and quality, implementing our first SLOs, and establishing a weekly meeting to review our SLOs and other reliability-related dashboards, fostering a culture of \textbf{continuous improvement}.
\smallskip

Spotify, I was responsible for \textbf{onboarding} and mentoring the new team replacing us, ensuring their success.
\smallskip

My next journey at Spotify involved transforming the \textbf{Customer Support tooling} Product Area into a \textbf{platform}. For nine months, I led the effort to define and implement a new product as the first successful example of a platform for our product area. I led a team of senior engineers in an \textbf{RFC-driven approach}, establishing close relationships with stakeholders and designers to identify needs and progress through different \textbf{MVP phases} to launch a full product. I was directly responsible for coordinating and collaborating with stakeholders, negotiating requirements and timelines, defining roadmaps, and ensuring we met our commitments. My focus on \textbf{reliability}, \textbf{scalability}, and \textbf{extensibility} resulted in a platform that easily onboards new lines of business and customer channels while streamlining integration with third-party vendors used by Customer Support.
\smallskip

After leading the initial definition, design, and launch of the product, I took ownership of the \textbf{frontend} component and mentored my team in frontend technologies while actively contributing to other areas. One initiative I led during this period was reducing the number of \textbf{PagerDuty alerts} from our legacy systems. I implemented a \textbf{data-driven approach} by creating metrics on alert frequency, sources, false positive rates, and categories. Subsequently, I guided the team in improving processes and enhancing our quality culture by conducting \textbf{post-mortems} for incidents and reducing alert noise. We successfully reduced PagerDuty alerts by over 80\%.
\smallskip

My latest focus is on the \textbf{reliability working group} of our product area, of which I am a founding member. Our first initiative, which has become a product area-wide effort with TSG approval, was to generate our first-ever \textbf{monthly availability report} for our product area.
